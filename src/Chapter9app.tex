\begin{problem}{9.1a}
You forecast an alpha of 2 percent for stocks that have E/P above the benchmark average \emph{and} IBES growth above the benchmark average. On average, what must your alpha forecasts be for stocks that do not satisfy these two criteria? If you assume an alpha of zero for stocks which have either above-average E/P or above-average IBES growth, but not both, what is your average alpha for stocks with E/P and IBES growth both below average?
\end{problem}

\begin{proof}[Solution]
Assume that the following groups all have equal capitalisation:
\begin{enumerate}
\item stocks with E/P above the benchmark average and IBES growth above the benchmark average,\label{above_above}
\item stocks with E/P below the benchmark average and IBES growth above the benchmark average,\label{below_above}
\item stocks with E/P above the benchmark average and IBES growth below the benchmark average,\label{above_below}
\item stocks with E/P below the benchmark average and IBES growth below the benchmark average.\label{below_below}
\end{enumerate}
We forecast an alpha of 2 percent for stocks in group~\ref{above_above}. If this forecast is weighted with the forecasts for the other three groups, it must balance out to zero. Hence, on average, for stocks in groups \ref{below_above}, \ref{above_below}, and \ref{below_below} our alpha forecasts must be $-2/3$ percent, so that $(3 \times -2/3) + 2 = 0$.

If, furthermore, we assume an alpha of zero for stocks in groups \ref{below_above} and \ref{above_below}, then our average alpha forecast for stocks in group \ref{below_below} must be $-2$, so that $2 + 0 + 0 - 2 = 0$. 
\end{proof}