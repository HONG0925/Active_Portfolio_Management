
\begin{problem}{4a.1}
Derive the benchmark timing result: \[\beta_{PA} = \frac{\Delta f_{B}}{\mu_{B}}\]
\end{problem}

\begin{proof}[Solution]
The question is a little terse. What it wants us to derive is equation (4.14), which says that for the portfolio $P$ which has the highest risk-adjusted return, we have \[\beta_{P} = 1 + \frac{\Delta f_{B}}{\mu_{B}}.\] It can be seen that this is equivalent to the equation in the question.

The portfolio $P$ is the portfolio which maximises the expected utility
\begin{align*}
U[P] &= f_{P} - \lambda_{T} \cdot \sigma_{P}^{2}. & (4.11)
\end{align*}
As per the discussion on p.98, the portfolio $P$ will be a mixture of $Q$ and $F$. Hence let the holdings of portfolio $P$ in risky assets be $\gamma \mathbf{h}_{Q}$. Here $\gamma$ is therefore simply the fraction of portfolio $P$ invested in portfolio $Q$, with the remainder as cash.

We wish to optimise $U[P]$ with respect to $\gamma$ and so we expand $U[P]$ in those terms:
\begin{align*}
U[P] &= f_{P} - \lambda_{T} \cdot \sigma_{P}^{2}\\
&= \gamma \cdot f_{Q} - \lambda_{T} \cdot \gamma^{2} \cdot \sigma_{Q}^{2}.
\end{align*}
We set the derivative $\frac{dU}{d\gamma}$ equal to zero, obtaining
\begin{align*}
&f_{Q} - 2 \lambda_{T} \cdot \gamma \cdot \sigma_{Q}^{2} = 0 \\
\therefore \quad &\gamma = \frac{f_{Q}}{2 \cdot \lambda_{T} \cdot \sigma_{Q}^{2}}. 
\end{align*}
The holdings of portfolio $P$ are therefore \[\frac{f_{Q}}{2 \cdot \lambda_{T} \cdot \sigma_{Q}^{2}}\mathbf{h}_{Q},\] as stated in the footnote on p.98.

We can now compute $\beta_{P}$:
\begin{align*}
\beta_{P} &= \frac{f_{Q}}{2 \cdot \lambda_{T} \cdot \sigma_{Q}^{2}}\beta_{Q}\\
&= \frac{f_{Q}}{2 \cdot \lambda_{T} \cdot \sigma_{Q}^{2}} \frac{f_{B}\sigma_{Q}^{2}}{f_{Q}\sigma_{B}^{2}} & \text{ by (2A.37)}\\
&= \frac{f_{B}}{2 \cdot \lambda \cdot \sigma_{B}^{2}}.
\end{align*}
This is equation (4.13). We then apply equation (4.12), which states that \[\lambda_{T} = \frac{\mu_{B}}{2 \cdot \sigma_{B}^{2}}.\] This gives
\begin{align*}
\beta_{P} &= \frac{f_{B}}{2 \cdot \lambda \cdot \sigma_{B}^{2}}\\
&= \frac{f_{B}}{\mu_{B}}\\
&= 1 + \frac{\Delta f_{B}}{\mu_{B}}.
\end{align*}
Hence \[\beta_{PA} = \frac{\Delta f_{B}}{\mu_{B}},\] as desired.
\end{proof}
