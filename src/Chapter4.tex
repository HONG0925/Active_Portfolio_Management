\begin{problem}{4.1}
 Assume a risk-free rate of 6 percent, a benchmark expected excess return of 6.5 percent, and a long range benchmark expected excess return of 6 percent. Given that McDonald's has a beta of 1.07 and an expected total return of 15 percent, separate its expected return into (a) time premium (b) risk premium (c) exceptional benchmark return (d) alpha (e) consensus expected return (f) expected excess return (g) exceptional expected return. (h) What is the sum of the consensus expected return and the exceptional expected return?
\end{problem}

\begin{proof}[Solution]
 From eq (4.7) we have the total expected return for stock $n$ broken into components as
 \begin{equation*}
  E\{R_{n}\}=1+i_{F}+\beta_{n}\mu_{B}+\beta_{n}\Delta f_{B}+\alpha_{n}
 \end{equation*}
 
 For McDonald's stock:
 \begin{enumerate}[label=(\alph*)]
  \item{The time premium is just the risk free rate, $i_{F}=6\%$.}
  \item{The risk premium is $\beta_{McDonald's}\mu_{B}=1.07\cdot 6\%=6.42\%$ where $\mu_{B}$ is the long range expected excess return of the benchmark.}
  \item{The exceptional benchmark return is $\beta_{McDonald's}\Delta f_{B}=1.07\cdot(6.5-6)\%=0.535\%$ where $\Delta f_{B}$ is the difference between the (immediate) expected excess return of the benchmark and the long run expected excess return of the benchmark}
  \item{Alpha can be found by solving the above equation and plugging in all of the values. We have
	\begin{align*}
	 \alpha_{McDonald's}&=E\{R_{McDonald's}\}-1-i_{F}-\beta_{n}\mu_{B}-\beta_{n}\Delta f_{B} \\
			    &=1.15-1-0.06-0.0642-0.00535\\
			    &=0.02045\\
	\end{align*}
	}
  \item{The consensus expected return is just $\beta_{McDonald's}\cdot\mu_{B}=6.42\%$}
  \item{The expected excess return is $f_{McDonald's}=E\{R_{McDonald's}\}-1-i_{F}=1.15-1-0.06=0.09$ or 9 percent}
  \item{The exceptional expected return is $f_{McDonald's}-\beta_{McDonald's}\mu_{B}=0.09-0.0642=0.0258$ or 2.58 percent.}
  \item{The sum of the consensus expected return and the exceptional return is $6.42\% + 2.58\%=9\%$ which is the expected excess return.}
 \end{enumerate}
\end{proof}

\begin{problem}{4.2}
 Suppose the benchmark is not the market, and the CAPM holds. How will the CAPM expected returns split into the categories suggested in this chapter?
\end{problem}

\begin{proof}[Solution]
 The CAPM expected excess returns of stock $n$ are equal to $\beta_{n}^{M}\mu_{M}$ where $\mu_{M}$ is the expected excess return of the market. Using eq (4.7), we can set 
 \begin{equation*}
  E\{R_{n}\}=1+i_{F}+\beta_{n}^{M}\mu_{M}=1+i_{F}+\beta_{n}^{B}\mu_{B}+\beta_{n}^{B}\Delta f_{B}+\alpha_{n}.
 \end{equation*}
 Here the superscripts indicate the market (M) and benchmark (B). For example $\beta_{n}^{B}$ is the beta of stock $n$ with respect to the benchmark while $\beta_{n}^{M}$ is the beta with respect to the market.
 
 If we suppose that the CAPM holds and that the benchmark is not the market, then we have $\mu_{B} = \beta_{B}^{M}\mu_{M}$. We also have that $\Delta f_{B} = 0$, since the near future expected benchmark return $f_{B}$ is also equal to $\beta_{B}^{M}\mu_{M}$. Hence \[E\{R_{n}\}=1+i_{F}+\beta_{n}^{M}\mu_{M}=1+i_{F}+\beta_{n}^{B}\beta_{B}^{M}\mu_{M}+\alpha_{n}.\] Therefore, the CAPM expected returns will split as follows into the categories suggested in this chapter.
\begin{itemize}
\item The time premium will still be equal to $i_{F}$.
\item The risk premium will equal $\beta_{n}^{B}\beta_{B}^{M}\mu_{M}$, which is also equal to the consensus expected return.
\item The exceptional benchmark return will be zero.
\item Alpha will be $(\beta_{B}^{M} - \beta_{n}^{B}\beta_{B}^{M})\mu_{M}$.
\item The expected excess return will be $\beta_{n}^{M}\mu_{M}$.
\item The exceptional expected return will be equal to the alpha, $(\beta_{B}^{M} - \beta_{n}^{B}\beta_{B}^{M})\mu_{M}$.
\end{itemize}
\end{proof}

\begin{problem}{4.3}
 Given a benchmark risk of 20 percent and a portfolio risk of 21 percent, and assuming a portfolio beta of 1, what is the portfolio's residual risk? What is its active risk? How does this compare to the difference between the portfolio risk and the benchmark risk?
\end{problem}

\begin{proof}[Solution]
 From Eq. (3.13), the variance of a stock $n$ is given by
 \begin{equation*}
  \sigma_{n}^{2}=\beta_{n}^{2}\sigma_{B}^{2} + \omega_{n}^{2}
 \end{equation*}
 so the variance of the portfolio $\bm{h_{P}}$ is given by
 \begin{align*}
  \bm{h_{P}^{T}}\cdot\bm{\sigma^{2}}&=\bm{h_{P}^{T}}\cdot\bm{\beta^{2}}\sigma_{B}^{2} + \bm{h_{P}^{T}}\cdot\bm{\omega^{2}}\\
  \sigma_{P}^{2}&=\beta_{P}^{2}\sigma_{B}^{2}+\omega_{P}^{2}
 \end{align*}
 where $\bm{\beta^{2}}$ is the vector of stock betas squared and $\bm{\omega^{2}}$ is the vector of stock residual returns squared. To find the portfolio's residual risk, we can solve for $\omega_{P}$ as.
 \begin{align*}
  \omega_{P} &= \sqrt{\sigma_{P}^{2}-\beta_{P}^{2}\sigma_{B}^{2}} \\
	     &= \sqrt{(21\%)^{2} - 1^{2}\times (20\%)^{2}} \\
	     &= 6.40\%
 \end{align*}
 The active risk is given by eq (4.20) as 
 \begin{equation*}
  \psi_{P}=\sqrt{\omega_{P}^{2}+\beta_{PA}^{2}\cdot\sigma_{B}^{2}}
 \end{equation*}
 The active beta is given by $\beta_{PA} = \beta_{P}-\beta_{B} = 1 - 1 = 0$. Hence, the active risk is equal to the residual risk at 6.40\%. The active risk is 6.4 percent compared to the difference of risk between the portfolio and the benchmark of 1 percent. The active risk is much larger than the simple difference in portfolio and benchmark risks.
 
\end{proof}


\begin{problem}{4.4}
 Investor A manages total return and risk ($f_{P}-\lambda_{T}\cdot\sigma_{P}^{2}$) with risk aversion $\lambda_{T} = 0.0075$. Investor B manages residual risk and return ($\alpha_{P}-\lambda_{R}\cdot\omega_{P}^{2}$), with risk aversion $\lambda_{R}=0.075$ (moderate to aggressive). They each can choose between two portfolios:
 \begin{align*}
  f_{1}&=10\%\\
  \sigma_{1}&=20.22\%\\
  f_{2}&=16\%\\
  \sigma_{2}&=25\%
 \end{align*}
 Both portfolios have $\beta=1$. Furthermore,
 \begin{align*}
  f_{B}&=6\%\\
  \sigma_{B}&=20\%
 \end{align*}
 Which portfolio will A prefer? Which portfolio will B prefer? (\textit{Hint:} First calculate expected residual return and residual risk for the two portfolios.)

\end{problem}

\begin{proof}[Solution]
 The residual returns are
 \begin{align*}
  \alpha_{1} &= f_{1} - \beta_{1}f_{B} \\
	     &= 10\%-1\times 6\%\\
	     &= 4\% \\
  \alpha_{2} &= f_{2} - \beta_{2}f_{B} \\
	     &= 16\%-1\times 6\%\\
	     &= 10\% 
 \end{align*}
 The residual risks are
 \begin{align*}
  \omega_{1} &= \sqrt{\sigma_{1}^{2} - \beta_{1}\sigma_{B}^{2}} \\
	     &= \sqrt{20.22 \% ^{2} - 1\times 20 \% ^2} \\
	     &= 2.975 \%\\
  \omega_{2} &= \sqrt{\sigma_{2}^{2} - \beta_{2}\sigma_{B}^{2}} \\
	     &= \sqrt{25 \% ^{2} - 1\times 20 \% ^2} \\
	     &= 15 \%
 \end{align*}
 Investor A will prefer the portfolio with maximum $f_{P}-\lambda_{T}\cdot\sigma_{P}^{2}$ (highest utility). We have
 \begin{align*}
  f_{1}-\lambda_{T}\cdot\sigma_{1}^{2} &= 10\% - 0.0075 \times 20.22^{2} \% \\
				       &= 10\% - 3.07\% \\
				       &= 6.93\%\\
  f_{2}-\lambda_{T}\cdot\sigma_{2}^{2} &= 16\% - 0.0075 \times 25^{2} \% \\
				       &= 16\% - 4.69\% \\
				       &= 11.31\%       
 \end{align*}
 so that investor A will prefer portfolio 2. On the other hand, investor B while refer the portfolio with maximum $\alpha_{P}-\lambda_{R}\cdot\omega_{P}^{2}$. We have
 \begin{align*}
  \alpha_{1}-\lambda_{R}\cdot\omega_{1}^{2} &= 4\% - 0.075 \times 2.975^{2} \% \\
				       &= 4\% - 0.65\% \\
				       &= 3.35\%\\
  \alpha_{2}-\lambda_{R}\cdot\omega_{2}^{2} &= 10\% - 0.075 \times 15^{2} \% \\
				       &= 10\% - 16.88\% \\
				       &= -6.88\%
 \end{align*}
 so that investor B will prefer portfolio 1.

\end{proof}

\begin{problem}{4.5}
 Assume that you are a mean/variance investor with total risk aversion of 0.0075. If a portfolio has an expected excess return of 6 percent and risk of 20 percent, what is your \textit{certainty equivalent return}, the certain expected excess return that you would fairly trade for this portfolio.
\end{problem}

\begin{proof}[Solution]
 The certainty equivalent return would be equal to the utility (see p 121) $f_{P}-\lambda_{T}\cdot\sigma_{P}^{2}$. We have
 \begin{align*}
  f_{P}-\lambda_{T}\cdot\sigma_{P}^{2}&=6\%-.0075\cdot 20^{2}\% \\
				      &=6\%-3\%\\
				      &=3\%
 \end{align*}

\end{proof}

