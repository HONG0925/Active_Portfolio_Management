
\begin{problem}{2a.1}
Show that $\beta_{C} = \frac{\sigma_{C}^{2}}{\sigma_{M}^{2}}$. Since portfolio $C$ is the minimum-variance portfolio, this relationship implies that $\beta_{C} \leqslant 1$, with $\beta_{C} = 1$ only if the market is the minimum-variance portfolio.
\end{problem}

\begin{proof}[Solution]
Since $\beta_{C}$ is defined relative to the market portfolio, we have that the characteristic of the market portfolio is $\mathbf{\beta}$ by (2A.19). Hence $\beta_{C}$ is equal to the exposure of portfolio $C$ to the characteristic of the market portfolio. Then, by (2A.4), we have \[e_{M}\sigma_{C}^{2} = \beta_{C}\sigma_{M}^{2}.\] We must have $e_{M} = 1$, since the balance of borrowing and lending in the market portfolio must balance out. Hence \[\beta_{C} = \frac{\sigma_{C}^{2}}{\sigma_{M}^{2}}.\]
\end{proof}

\begin{problem}{2a.2}
Show that $f_{Q} = f_{C} + \frac{\sigma_{C}^{2}}{\kappa f_{C}}$, i.e., $\kappa = \frac{\sigma_{C}^{2}}{f_{C}(f_{Q} - f_{C})}$
\end{problem}

\begin{proof}[Solution]
We begin with (2A.35):
\[\frac{f_{C}}{\sigma_{C}^{2}} = \frac{f_{Q}}{\sigma_{Q}^{2}}.\]
We then deduce
\begin{align*}
&f_{C}\sigma_{Q}^{2} = f_{Q}\sigma_{C}^{2}\\
\therefore \quad &f_{C}(\sigma_{Q}^{2} - \sigma_{C}^{2}) = \sigma_{C}^{2}(f_{Q} - f_{C})\\
\therefore \quad &\frac{(\sigma_{Q}^{2} - \sigma_{C}^{2})}{(f_{Q} - f_{C})^{2}} = \frac{\sigma_{C}^{2}}{f_{C}(f_{Q} - f_{C})}.
\end{align*}
This then gives us that \[\kappa = \frac{\sigma_{C}^{2}}{f_{C}(f_{Q} - f_{C})},\] as desired.
\end{proof}

\begin{problem}{2a.3}
What is the ``characteristic'' associated with the MMI portfolio? How would you find it?
\end{problem}

\begin{proof}[Solution]
By Proposition 1.3, the characteristic associated with a portfolio is the vector of betas of all assets with respect to the portfolio. This then applies to the MMI portfolio.

To find the characteristic, one could calculate these betas using regression.
\end{proof}

\begin{problem}{2a.4}
Prove that the fully invested portfolio that maximizes $f_{P} - \lambda\sigma_{P}^{2}$ has expected excess return $f^{\ast} = f_{C} + \frac{1}{2\lambda\kappa}$.
\end{problem}

\begin{proof}[Solution]
To start, we note the following facts, which we will need to use in this question, as well as later questions.
\begin{align*}
\mathbf{h}_{C} &= \frac{\mathbf{V}^{-1}\mathbf{e}}{\mathbf{e}^{T}\mathbf{V}^{-1}\mathbf{e}}, & (2A.14)\\
\sigma_{C}^{2} &= \frac{1}{\mathbf{e}^{T}\mathbf{V}^{-1}\mathbf{e}}, & (2A.15).
\end{align*}
We also have \[h_{Q} = \frac{\mathbf{h}_{q}}{e_{q}} = \frac{\mathbf{V}^{-1}\mathbf{f}}{\mathbf{f}^{T}\mathbf{V}^{-1}\mathbf{f}}\frac{\mathbf{f}^{T}\mathbf{V}^{-1}\mathbf{f}}{\mathbf{e}^{T}\mathbf{V}^{-1}\mathbf{f}} = \frac{\mathbf{V}^{-1}\mathbf{f}}{\mathbf{e}^{T}\mathbf{V}^{-1}\mathbf{f}},\] where we draw upon (2A.23) and Proposition 3. We also note \[\sigma_{Q}^{2} = \frac{f_{Q}\sigma_{C}^{2}}{f_{C}} = \frac{\mathbf{f}^{T}\mathbf{V}^{-1}\mathbf{f}}{\mathbf{e}^{T}\mathbf{V}^{-1}\mathbf{f}}\frac{1}{\mathbf{e}^{T}\mathbf{V}^{-1}\mathbf{e}}\frac{\mathbf{e}^{T}\mathbf{V}^{-1}\mathbf{e}}{\mathbf{f}^{T}\mathbf{V}^{-1}\mathbf{e}} = \frac{\mathbf{f}^{T}\mathbf{V}^{-1}\mathbf{f}}{(\mathbf{e}^{T}\mathbf{V}^{-1}\mathbf{f})^{2}},\] drawing upon (2A.35). Finally, note that $\mathbf{f}^{T}\mathbf{V}^{-1}\mathbf{e} = \mathbf{e}^{T}\mathbf{V}^{-1}\mathbf{f}$ because $\mathbf{V}$ is a symmetric matrix, and so $\mathbf{V}^{-1}$ is symmetric too.

In this question we need to maximise \[\mathbf{h}^{T}\mathbf{f} - \lambda\mathbf{h}^{T}\mathbf{V}\mathbf{h}\] subject to the constraint \[\mathbf{e}^{T}\mathbf{h} = 1.\] We use the method of Lagrange multipliers, considering the function \[\mathbf{h}^{T}\mathbf{f} - \lambda\mathbf{h}^{T}\mathbf{V}\mathbf{h} - \theta(\mathbf{e}^{T}\mathbf{h} - 1)\] and obtaining the equations
\begin{align*}
\mathbf{f} - 2\lambda\mathbf{V}\mathbf{h} - \theta\mathbf{e} &= 0, \\
\mathbf{e}^{T}\mathbf{h} &= 1.
\end{align*}
By rearranging the first equation, we obtain that \[\mathbf{h} = \frac{1}{2\lambda}\mathbf{V}^{-1}(\mathbf{f} - \theta\mathbf{e}).\] We then substitute this into the second equation, which gives us that \[\frac{1}{2\lambda}\mathbf{e}^{T}\mathbf{V}^{-1}(\mathbf{f} - \theta\mathbf{e}) = 1.\] We can now solve for $\theta$:
\begin{align*}
&\mathbf{e}^{T}\mathbf{V}^{-1}(\mathbf{f} - \theta\mathbf{e}) = 2\lambda \\
\therefore \quad &\theta\mathbf{e}^{T}\mathbf{V}^{-1}\mathbf{e} = \mathbf{e}^{T}\mathbf{V}^{-1}\mathbf{f} - 2\lambda \\
\therefore \quad &\theta = \frac{\mathbf{e}^{T}\mathbf{V}^{-1}\mathbf{f}}{\mathbf{e}^{T}\mathbf{V}^{-1}\mathbf{e}} - \frac{2\lambda}{\mathbf{e}^{T}\mathbf{V}^{-1}\mathbf{e}}.
\end{align*}
Hence, the holdings of the portfolio are
\begin{align*}
\mathbf{h} &= \frac{1}{2\lambda}\mathbf{V}^{-1}\left(\frac{\mathbf{e}^{T}\mathbf{V}^{-1}\mathbf{f}}{\mathbf{e}^{T}\mathbf{V}^{-1}\mathbf{e}} - \frac{2\lambda}{\mathbf{e}^{T}\mathbf{V}^{-1}\mathbf{e}}\mathbf{e}\right)\\
&= \frac{\mathbf{V}^{-1}\mathbf{e}}{\mathbf{e}^{T}\mathbf{V}^{-1}\mathbf{e}} + \frac{1}{2\lambda}\left(\mathbf{V}^{-1}\mathbf{f} - \frac{\mathbf{e}^{T}\mathbf{V}^{-1}\mathbf{f}}{\mathbf{e}^{T}\mathbf{V}^{-1}\mathbf{e}}\mathbf{V}^{-1}\mathbf{e}\right).
\end{align*}
From this we can compute the expected excess return of the portfolio, namely
\begin{align*}
\mathbf{f}^{T}\mathbf{h} &= \frac{\mathbf{f}^{T}\mathbf{V}^{-1}\mathbf{e}}{\mathbf{e}^{T}\mathbf{V}^{-1}\mathbf{e}} + \frac{1}{2\lambda}\left(\mathbf{f}^{T}\mathbf{V}^{-1}\mathbf{f} - \frac{(\mathbf{e}^{T}\mathbf{V}^{-1}\mathbf{f})^{2}}{\mathbf{e}^{T}\mathbf{V}^{-1}\mathbf{e}}\right) & \\
&= f_{C} + \frac{1}{2\lambda}\left(\mathbf{f}^{T}\mathbf{V}^{-1}\mathbf{f} - \frac{(\mathbf{e}^{T}\mathbf{V}^{-1}\mathbf{f})^{2}}{\mathbf{e}^{T}\mathbf{V}^{-1}\mathbf{e}}\right) & \\
&= f_{C} + \frac{1}{2\lambda}\frac{\left(\mathbf{f}^{T}\mathbf{V}^{-1}\mathbf{f} - \frac{(\mathbf{e}^{T}\mathbf{V}^{-1}\mathbf{f})^{2}}{\mathbf{e}^{T}\mathbf{V}^{-1}\mathbf{e}}\right)\left(\frac{\mathbf{f}^{T}\mathbf{V}^{-1}\mathbf{f}}{(\mathbf{e}^{T}\mathbf{V}^{-1}\mathbf{f})^{2}} - \frac{1}{\mathbf{e}^{T}\mathbf{V}^{-1}\mathbf{e}}\right)}{\frac{\mathbf{f}^{T}\mathbf{V}^{-1}\mathbf{f}}{(\mathbf{e}^{T}\mathbf{V}^{-1}\mathbf{f})^{2}} - \frac{1}{\mathbf{e}^{T}\mathbf{V}^{-1}\mathbf{e}}} &  \\
&= f_{C} + \frac{1}{2\lambda}\frac{\frac{(\mathbf{f}^{T}\mathbf{V}^{-1}\mathbf{f})^{2}}{(\mathbf{e}^{T}\mathbf{V}^{-1}\mathbf{f})^{2}} - 2\frac{\mathbf{f}^{T}\mathbf{V}^{-1}\mathbf{f}}{\mathbf{e}^{T}\mathbf{V}^{-1}\mathbf{e}} + \frac{(\mathbf{e}^{T}\mathbf{V}^{-1}\mathbf{f})^{2}}{(\mathbf{e}^{T}\mathbf{V}^{-1}\mathbf{e})^{2}}}{\frac{\mathbf{f}^{T}\mathbf{V}^{-1}\mathbf{f}}{(\mathbf{e}^{T}\mathbf{V}^{-1}\mathbf{f})^{2}} - \frac{1}{\mathbf{e}^{T}\mathbf{V}^{-1}\mathbf{e}}} & \\
&= f_{C} + \frac{1}{2\lambda}\frac{f_{Q}^{2} - 2f_{Q}f_{C} - f_{C}^{2}}{\sigma_{Q}^{2} - \sigma_{C}^{2}} & \\
&= f_{C} + \frac{1}{2\lambda}\frac{(f_{Q} - f_{C})^{2}}{\sigma_{Q}^{2} - \sigma_{C}^{2}} &\\
&= f_{C} + \frac{1}{2\lambda\kappa}, & 
\end{align*}
as desired.
\end{proof}

\begin{problem}{2a.5}
Prove that portfolio $Q$ is the optimal solution in Exercise 4 if $\lambda = \frac{f_{C}}{2\sigma_{C}^{2}} = \frac{f_{Q}}{2\sigma_{Q}^{2}}$.
\end{problem}

\begin{proof}[Solution]
By (2A.14) and (2A.15), \[\frac{f_{C}}{\sigma_{C}^{2}} = \mathbf{f}^{T}\mathbf{V}^{-1}\mathbf{e}.\] By the solution to the previous problem, we have \[\mathbf{h} = \frac{\mathbf{V}^{-1}\mathbf{e}}{\mathbf{e}^{T}\mathbf{V}^{-1}\mathbf{e}} + \frac{1}{2\lambda}\left(\mathbf{V}^{-1}\mathbf{f} - \frac{\mathbf{e}^{T}\mathbf{V}^{-1}\mathbf{f}}{\mathbf{e}^{T}\mathbf{V}^{-1}\mathbf{e}}\mathbf{V}^{-1}\mathbf{e}\right).\] Hence, if $\lambda =  \frac{f_{C}}{2\sigma_{C}^{2}}$, we have that
\begin{align*}
\mathbf{h} &= \frac{\mathbf{V}^{-1}\mathbf{e}}{\mathbf{e}^{T}\mathbf{V}^{-1}\mathbf{e}} + \frac{\mathbf{V}^{-1}\mathbf{f}}{\mathbf{f}^{T}\mathbf{V}^{-1}\mathbf{e}} - \frac{\mathbf{e}^{T}\mathbf{V}^{-1}\mathbf{f}}{\mathbf{e}^{T}\mathbf{V}^{-1}\mathbf{e}}\frac{\mathbf{V}^{-1}\mathbf{e}}{\mathbf{f}^{T}\mathbf{V}^{-1}\mathbf{e}} \\
&= \frac{\mathbf{V}^{-1}\mathbf{f}}{\mathbf{f}^{T}\mathbf{V}^{-1}\mathbf{e}},
\end{align*}
which are the holdings of portfolio $Q$, as we noted in the solution to the previous problem.
\end{proof}


\begin{problem}{2a.6}
Suppose portfolio $T$ is on the fully invested efficient frontier. Prove Eq. (2A.45), i.e., that there
exists a $w_{T}$ such that $h_T = w_T h_C + (1 – w_T )h_Q$.
\end{problem}

\begin{proof}[Solution]
Since $T$ is on the fully invested efficient frontier, $T$ has minimum risk amongst all portfolios with the same expected return $f_{P}$. Hence, to find $T$, we wish to minimise \[\frac{\mathbf{h}^{T}\mathbf{V}\mathbf{h}}{2}\] subject to the constraints
\begin{align*}
\mathbf{e}^{T}\mathbf{h} &= 1\\
\mathbf{f}^{T}\mathbf{h} &= f_{P}.
\end{align*}
As always, we use the method of Lagrange multipliers and so consider \[\frac{\mathbf{h}^{T}\mathbf{V}\mathbf{h}}{2} + \theta_{1}(\mathbf{e}^{T}\mathbf{h} - 1) + \theta_{2}(\mathbf{f}^{T}\mathbf{h} - f_{P}).\] We hence need to solve the simultaneous equations
\begin{align*}
\mathbf{V}\mathbf{h} + \theta_{1}\mathbf{e} + \theta_{2}\mathbf{f} &= 0, \\
\mathbf{e}^{T}\mathbf{h} &= 1, \\
\mathbf{f}^{T}\mathbf{h} &= f_{P},
\end{align*}
From the first equation we obtain that 
\begin{equation}\label{eq:6h}
\mathbf{h} = \mathbf{V}^{-1}(-\theta_{1}\mathbf{e} - \theta_{2}\mathbf{f}).
\end{equation}
Substituting this into the other two equations gives
\begin{align*}
-\theta_{1}\mathbf{e}^{T}\mathbf{V}^{-1}\mathbf{e} - \theta_{2}\mathbf{e}^{T}\mathbf{V}^{-1}\mathbf{f} &= 1,\\
-\theta_{1}\mathbf{f}^{T}\mathbf{V}^{-1}\mathbf{e} - \theta_{2}\mathbf{f}^{T}\mathbf{V}^{-1}\mathbf{f} &= f_{P}.
\end{align*}
We now have two simultaneous equations for $\theta_{1}$ and $\theta_{2}$, which we can solve as follows. We rearrange the first equation for $\theta_{1}$, obtaining
\[\theta_{1} = - \frac{\theta_{2}\mathbf{e}^{T}\mathbf{V}^{-1}\mathbf{f} + 1}{\mathbf{e}^{T}\mathbf{V}^{-1}\mathbf{e}}.\] We can then substitute this into the second equation to give \[\frac{(\theta_{2}\mathbf{e}^{T}\mathbf{V}^{-1}\mathbf{f} + 1)\mathbf{e}^{T}\mathbf{V}^{-1}\mathbf{f}}{\mathbf{e}^{T}\mathbf{V}^{-1}\mathbf{e}} - \theta_{2}\mathbf{f}^{T}\mathbf{V}^{-1}\mathbf{f} = f_{P}.\] We now solve this equation for $\theta_{2}$.
\begin{align*}
&\theta_{2}(\mathbf{e}^{T}\mathbf{V}^{-1}\mathbf{f})^{2} + \mathbf{e}^{T}\mathbf{V}^{-1}\mathbf{f} - \theta_{2}(\mathbf{f}^{T}\mathbf{V}^{-1}\mathbf{f})(\mathbf{e}^{T}\mathbf{V}^{-1}\mathbf{e}) = f_{P}\mathbf{e}^{T}\mathbf{V}^{-1}\mathbf{e} \\
\therefore \quad &((\mathbf{e}^{T}\mathbf{V}^{-1}\mathbf{f})^{2} - (\mathbf{f}^{T}\mathbf{V}^{-1}\mathbf{f})(\mathbf{e}^{T}\mathbf{V}^{-1}\mathbf{e}))\theta_{2} = f_{P}\mathbf{e}^{T}\mathbf{V}^{-1}\mathbf{e} - \mathbf{e}^{T}\mathbf{V}^{-1}\mathbf{f} \\
\therefore \quad &\theta_{2} = \frac{f_{P}\mathbf{e}^{T}\mathbf{V}^{-1}\mathbf{e} - \mathbf{e}^{T}\mathbf{V}^{-1}\mathbf{f}}{(\mathbf{e}^{T}\mathbf{V}^{-1}\mathbf{f})^{2} - (\mathbf{f}^{T}\mathbf{V}^{-1}\mathbf{f})(\mathbf{e}^{T}\mathbf{V}^{-1}\mathbf{e})}
\end{align*}
This gives $\theta_{1}$ via our expression for $\theta_{1}$ in terms of $\theta_{2}$.
\begin{align*}
\theta_{1} &= - \frac{\frac{f_{P}\mathbf{e}^{T}\mathbf{V}^{-1}\mathbf{e} - \mathbf{e}^{T}\mathbf{V}^{-1}\mathbf{f}}{(\mathbf{e}^{T}\mathbf{V}^{-1}\mathbf{f})^{2} - (\mathbf{f}^{T}\mathbf{V}^{-1}\mathbf{f})(\mathbf{e}^{T}\mathbf{V}^{-1}\mathbf{e})}\mathbf{e}^{T}\mathbf{V}^{-1}\mathbf{f} + 1}{\mathbf{e}^{T}\mathbf{V}^{-1}\mathbf{e}} \\
&= -\frac{(f_{P}\mathbf{e}^{T}\mathbf{V}^{-1}\mathbf{e} - \mathbf{e}^{T}\mathbf{V}^{-1}\mathbf{f})\mathbf{e}^{T}\mathbf{V}^{-1}\mathbf{f} + (\mathbf{e}^{T}\mathbf{V}^{-1}\mathbf{f})^{2} - (\mathbf{f}^{T}\mathbf{V}^{-1}\mathbf{f})(\mathbf{e}^{T}\mathbf{V}^{-1}\mathbf{e})}{\mathbf{e}^{T}\mathbf{V}^{-1}\mathbf{e}((\mathbf{e}^{T}\mathbf{V}^{-1}\mathbf{f})^{2} - (\mathbf{f}^{T}\mathbf{V}^{-1}\mathbf{f})(\mathbf{e}^{T}\mathbf{V}^{-1}\mathbf{e}))}\\
&= - \frac{f_{P}\mathbf{e}^{T}\mathbf{V}^{-1}\mathbf{f} - \mathbf{f}^{T}\mathbf{V}^{-1}\mathbf{f}}{(\mathbf{e}^{T}\mathbf{V}^{-1}\mathbf{f})^{2} - (\mathbf{f}^{T}\mathbf{V}^{-1}\mathbf{f})(\mathbf{e}^{T}\mathbf{V}^{-1}\mathbf{e})}
\end{align*}
Having solved for $\theta_{1}$ and $\theta_{2}$, we now substitute back into (\ref{eq:6h}) to find $\mathbf{h}$.
\begin{align*}
\mathbf{h} &= -\theta_{1}\mathbf{V}^{-1}\mathbf{e} - \theta_{2} \mathbf{V}^{-1}\mathbf{f}\\
&= \frac{f_{P}\mathbf{e}^{T}\mathbf{V}^{-1}\mathbf{f} - \mathbf{f}^{T}\mathbf{V}^{-1}\mathbf{f}}{(\mathbf{e}^{T}\mathbf{V}^{-1}\mathbf{f})^{2} - (\mathbf{f}^{T}\mathbf{V}^{-1}\mathbf{f})(\mathbf{e}^{T}\mathbf{V}^{-1}\mathbf{e})}\mathbf{V}^{-1}\mathbf{e} + \frac{\mathbf{e}^{T}\mathbf{V}^{-1}\mathbf{f} - f_{P}\mathbf{e}^{T}\mathbf{V}^{-1}\mathbf{e}}{(\mathbf{e}^{T}\mathbf{V}^{-1}\mathbf{f})^{2} - (\mathbf{f}^{T}\mathbf{V}^{-1}\mathbf{f})(\mathbf{e}^{T}\mathbf{V}^{-1}\mathbf{e})}\mathbf{V}^{-1}\mathbf{f}\\
&= \frac{f_{P}(\mathbf{e}^{T}\mathbf{V}^{-1}\mathbf{f})(\mathbf{e}^{T}\mathbf{V}^{-1}\mathbf{e}) - (\mathbf{f}^{T}\mathbf{V}^{-1}\mathbf{f})(\mathbf{e}^{T}\mathbf{V}^{-1}\mathbf{e})}{(\mathbf{e}^{T}\mathbf{V}^{-1}\mathbf{f})^{2} - (\mathbf{f}^{T}\mathbf{V}^{-1}\mathbf{f})(\mathbf{e}^{T}\mathbf{V}^{-1}\mathbf{e})}\frac{\mathbf{V}^{-1}\mathbf{e}}{\mathbf{e}^{T}\mathbf{V}^{-1}\mathbf{e}} + \frac{(\mathbf{e}^{T}\mathbf{V}^{-1}\mathbf{f})^{2} - f_{P}(\mathbf{e}^{T}\mathbf{V}^{-1}\mathbf{e})(\mathbf{e}^{T}\mathbf{V}^{-1}\mathbf{f})}{(\mathbf{e}^{T}\mathbf{V}^{-1}\mathbf{f})^{2} - (\mathbf{f}^{T}\mathbf{V}^{-1}\mathbf{f})(\mathbf{e}^{T}\mathbf{V}^{-1}\mathbf{e})}\frac{\mathbf{V}^{-1}\mathbf{f}}{\mathbf{e}^{T}\mathbf{V}^{-1}\mathbf{f}}\\
&= \frac{f_{P} - \frac{\mathbf{f}^{T}\mathbf{V}^{-1}\mathbf{f}}{\mathbf{e}^{T}\mathbf{V}^{-1}\mathbf{f}}}{\frac{\mathbf{e}^{T}\mathbf{V}^{-1}\mathbf{f}}{\mathbf{e}^{T}\mathbf{V}^{-1}\mathbf{e}} - \frac{\mathbf{f}^{T}\mathbf{V}^{-1}\mathbf{f}}{\mathbf{e}^{T}\mathbf{V}^{-1}\mathbf{f}}}h_{C} + \frac{\frac{\mathbf{e}^{T}\mathbf{V}^{-1}\mathbf{f}}{\mathbf{e}^{T}\mathbf{V}^{-1}\mathbf{e}} - f_{P}}{\frac{\mathbf{e}^{T}\mathbf{V}^{-1}\mathbf{f}}{\mathbf{e}^{T}\mathbf{V}^{-1}\mathbf{e}} - \frac{\mathbf{f}^{T}\mathbf{V}^{-1}\mathbf{f}}{\mathbf{e}^{T}\mathbf{V}^{-1}\mathbf{f}}}h_{Q}\\
&= \frac{f_{P} - f_{Q}}{f_{C} - f_{Q}}h_{C} + \frac{f_{C} - f_{P}}{f_{C} - f_{Q}}h_{Q}\\
&= \frac{f_{Q} - f_{P}}{f_{Q} - f_{C}}h_{C} + \frac{f_{P} - f_{C}}{f_{Q} - f_{C}}h_{C}.
\end{align*}
We hence recover equation (2A.45).
\end{proof}


\begin{problem}{2a.7}
If $T$ is fully invested and efficient and $T \neq C$, prove that there exists a fully invested efficient portfolio $T^{\ast}$ such that $\cov{r_{T}, r_{T^{\ast}}} = 0$.
\end{problem}

\begin{proof}[Solution]
By the previous problem, we know that we must be able to write
\begin{align*}
h_{T} &= w_{T}h_{C} + (1 - w_{T})h_{Q}\\
h_{T^{\ast}} &= w_{T^{\ast}}h_{C} + (1 - w_{T^{\ast}})h_{Q},
\end{align*}
since $T$ and $T^{\ast}$ are fully invested efficient portfolios. Hence, we have
\begin{align*}
r_{T} &= w_{T}r_{C} + (1 - w_{T})r_{Q}\\
r_{T^{\ast}} &= w_{T^{\ast}}r_{C} + (1 - w_{T^{\ast}})r_{Q}.
\end{align*}
Our approach is to use these equations to compute $\cov{r_{T}, r_{T^{\ast}}}$ in terms of $w_{T}$, $w_{T^{\ast}}$, and constants. Then, by setting this covariance equal to zero, we can solve for $w_{T^{\ast}}$ in terms of $w_{T}$. This then gives us the desired portfolio $T^{\ast}$ which is uncorrelated with $T$. Thus
\begin{align*}
\cov{r_{T}, r_{T^{\ast}}} &= \cov{w_{T}r_{C} + (1 - w_{T})r_{Q}, w_{T^{\ast}}r_{C} + (1 - w_{T^{\ast}})r_{Q}} \\
&= w_{T}w_{T^{\ast}}\sigma_{C}^{2} + (1 - w_{T})(1 - w_{T^{\ast}})\sigma_{Q}^{2} + ((1 - w_{T})w_{T^{\ast}} + (1 - w_{T^{\ast}})w_{T})\sigma_{C, Q}\\
&= w_{T}w_{T^{\ast}}\sigma_{C}^{2} + (1 - w_{T})(1 - w_{T^{\ast}})\frac{f_{Q}}{f_{C}}\sigma_{C}^{2} + ((1 - w_{T})w_{T^{\ast}} + (1 - w_{T^{\ast}})w_{T})\sigma_{C}^{2} & \because (2A.17), (2A.35)\\
&= \sigma_{C}^{2}\left(w_{T}w_{T^{\ast}} + \frac{f_{Q}}{f_{C}} - \frac{f_{Q}}{f_{C}}w_{T} - \frac{f_{Q}}{f_{C}}w_{T^{\ast}} + \frac{f_{Q}}{f_{C}}w_{T}w_{T^{\ast}} + w_{T^{\ast}} + w_{T} - 2w_{T}w_{T^{\ast}}\right) \\
&= \sigma_{C}^{2}\left[ \left(\frac{f_{Q}}{f_{C}} - 1 \right)w_{T}w_{T^{\ast}} + \left(1 - \frac{f_{Q}}{f_{C}}\right)w_{T} + \left(1 - \frac{f_{Q}}{f_{C}}\right)w_{T^{\ast}} + \frac{f_{Q}}{f_{C}}\right].
\end{align*}
Since we desire $\cov{r_{T}, r_{T^{\ast}}} = 0$, we must have
\begin{align*}
&\left(\frac{f_{Q}}{f_{C}} - 1 \right)w_{T}w_{T^{\ast}} + \left(1 - \frac{f_{Q}}{f_{C}}\right)w_{T} + \left(1 - \frac{f_{Q}}{f_{C}}\right)w_{T^{\ast}} + \frac{f_{Q}}{f_{C}} = 0\\
\iff \quad &\left[ \left(\frac{f_{Q}}{f_{C}} - 1\right) w_{T} + \left(1 - \frac{f_{Q}}{f_{C}}\right)\right]w_{T^{\ast}} = \left(\frac{f_{Q}}{f_{C}} - 1\right)w_{T} - \frac{f_{Q}}{f_{C}}\\
\iff \quad &w_{T^{\ast}} = \frac{\left(\frac{f_{Q}}{f_{C}} - 1\right)w_{T} -\frac{f_{Q}}{f_{C}}}{\left(\frac{f_{Q}}{f_{C}} - 1 \right)w_{T} + \left(1 - \frac{f_{Q}}{f_{C}}\right)}.
\end{align*}
Thus, if we set $w_{T^{\ast}}$ as above, we have $\cov{r_{T}, r_{T^{\ast}}} = 0$. This establishes the existence of such a portfolio $T^{\ast}$.

Note that we must assume $T \neq C$, so that $w_{T} \neq 1$ and the denominator of the fraction is non-zero.
\end{proof}


\begin{problem}{2a.8}
For any $T \neq C$ on the efficient frontier and any fully invested portfolio $P$, show that we can write \[E\{r_{P}\} = E\{r_{T^{\ast}}\} + E\{r_{T} - r_{T^{\ast}}\}\frac{\mathrm{Cov}\{r_{P}, r_{T}\}}{\mathrm{Var}\{r_{T}\}}\] where $T^{\ast}$ is the fully invested efficient portfolio that is uncorrelated with $T$.
\end{problem}

\begin{proof}[Solution]
We proceed as follows. We must assume that $T \neq C$ so that we can apply Problem 2a.7.
\begin{align*}
E\{r_{T^{\ast}}\} &+ E\{r_{T} - r_{T^{\ast}}\}\frac{\cov{r_{P, r_{T}}}}{\var{r_{T}}}\}\\
&= f_{T^{\ast}} + (f_{T} - f_{T^{\ast}})\frac{\cov{r_{P} - r_{T^{\ast}}, r_{T}}}{\cov{r_{T} - r_{T^{\ast}}, r_{T}}} ~ \because ~ T \text{ and } T^{ast} \text{ are uncorrelated}\\
&= f_{T^{\ast}} + (f_{T} - f_{T^{\ast}})\frac{\cov{r_{P} - r_{T^{\ast}}, w_{T}r_{C} + (1 - w_{T})r_{Q}}}{\cov{r_{T} - r_{T^{\ast}}, w_{T}r_{C} + (1 - w_{T})r_{Q}}} \\
&= f_{T^{\ast}} + (f_{T} - f_{T^{\ast}})\frac{w_{T}(\cov{r_{P}, r_{C}} - \cov{r_{T^{\ast}}, r_{C}}) + (1 - w_{T})(\cov{r_{P}, r_{Q}} - \cov{r_{T^{\ast}, r_{Q}}})}{w_{T}(\cov{r_{T}, r_{C}} - \cov{r_{T^{\ast}}, r_{C}}) + (1 - w_{T})(\cov{r_{T}, r_{Q}} - \cov{r_{T^{\ast}, r_{Q}}})}\\
&= f_{T^{\ast}} + (f_{T} - f_{T^{\ast}})\frac{w_{T}(e_{P}\sigma_{C}^{2} - e_{T^{\ast}}\sigma_{C}^{2}) + (1 - w_{T})(e_{q}f_{P}\sigma_{Q}^{2} - e_{q}f_{T^{\ast}}\sigma_{Q}^{2})}{w_{T}(e_{T}\sigma_{C}^{2} - e_{T^{\ast}}\sigma_{C}^{2}) + (1 - w_{T})(e_{q}f_{T}\sigma_{Q}^{2} - e_{q}f_{T^{\ast}}\sigma_{Q}^{2})} ~ \because ~ (2A.4)\\
&= f_{T^{\ast}} + (f_{T} - f_{T^{\ast}})\frac{(1 - w_{T})e_{q}\sigma_{Q}^{2}(f_{P} - f_{T^{\ast}})}{(1 - w_{T})e_{q}\sigma_{Q}^{2}(f_{T} - f_{T^{\ast}})} ~ \because ~ \text{The portfolios are all fully invested}\\
&= f_{T^{\ast}} + (f_{P} - f_{T^{\ast}})\\
&= f_{P}.
\end{align*}
Note that $1 - w_{T} \neq 0$ since $T \neq C$.
\end{proof}


\begin{problem}{2a.9}
If $P$ is any fully invested portfolio and $T$ is the efficient fully invested portfolio with the same expected returns as $P$, $\mu_{P} = \mu_{T}$, we can always write the returns to $P$ as $r_{P} = r_{C} + (r_{T} - r_{C}) + (r_{P} - r_{T})$. Prove that these three components of return are uncorrelated. We can interpret the risks associated with these three components as the cost of full investment, $\mathrm{Var}\{r_{C}\}$; the cost of expected return $\mu_{P} - \mu_{C}$, $\mathrm{Var}\{r_{T} - r_{C}\}$; and the diversifiable cost, $\mathrm{Var}\{r_{P} - r_{T}\}$.
\end{problem}

\begin{proof}[Solution]
There are three pairs of components that we need to show are uncorrelated and we address each in turn.
\begin{align*}
\mathrm{Cov}\{r_{C}, r_{T} - r_{C}\} &= \cov{r_{C}, r_{T}} - \cov{r_{C}, r_{C}}\\
&= e_{T}\sigma_{C}^{2} - \sigma_{C}^{2} & (2A.4)\\
&= 0.
\end{align*}
Here the last line follows from the fact that $T$ is fully invested. We next consider the following.
\begin{align*}
\cov{r_{C}, r_{P} - r_{T}} &= \cov{r_{C}, r_{P}} - \cov{r_{C}, r_{T}} \\
&= e_{P}\sigma_{C}^{2} - e_{T}\sigma_{C}^{2} & (2A.4)\\
&= 0.
\end{align*}
Again, the last line follows from the fact that $T$ and $P$ are both fully invested. We now consider the last pair of components.
\begin{align*}
\cov{r_{T} - r_{C}, r_{P} - r_{T}} &= \cov{r_{T}, r_{P} - r_{T}} - \cov{r_{C}, r_{P} - r_{T}}\\
&= \cov{r_{T}, r_{P} - r_{T}},
\end{align*}
by above. By Problem 2.a8, we have
\begin{align*}
&& E\{r_{P}\} &= E\{r_{T^{\ast}}\} + E\{r_{T} - r_{T^{\ast}}\}\frac{\cov{r_{P}, r_{T}}}{\var{r_{T}}}\\
& \therefore & E\{r_{T}\}\var{r_{T}} &= \var{r_{T}}E\{r_{T^{\ast}}\} + E\{r_{T} - r_{T^{\ast}}\}\cov{r_{P}, r_{T}} & \because E\{r_{T}\} = E\{r_{P}\}\\
& \therefore & E\{r_{T} - r_{T^{\ast}}\}\var{r_{T}} &= E\{r_{T} - r_{T^{\ast}}\}\cov{r_{P}, r_{T}}\\
& \therefore & \var{r_{T}} &= \cov{r_{P}, r_{T}} \\
& \therefore & \cov{r_{T}, r_{P} - r_{T}} &= 0.
\end{align*}
This completes the proof.
\end{proof}

